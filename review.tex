\title{Review of \say{The matroid of a graphing}}
\date{\today}


\newcommand\entry[2]{\item Page {#1}:  \say{#2} $\rightarrow$}
\def\shuffle{\sqcup\mathchoice{\mkern-7mu}{\mkern-7mu}{\mkern-3.2mu}{\mkern-3.8mu}\sqcup}

\documentclass[12pt]{article}
\usepackage{dirtytalk}
\usepackage{amsmath}
\usepackage{amssymb}
\begin{document}
\maketitle




\section*{Outline}

The paper introduces a notion of cycle matriod for a \say{Benjamini - Schramm} limiting object on graphs, generalising the notion of cycle matroid.
It tacitly aims at kickstarting a new limit theory on matroids, presenting the proof of fundamental claims (for instance, the author proves that the  normalized rank functions of a sequence of graphs converges to the normalized rank function of the Benjamini Schramm limit) and disclosing some conjectures ( e.g. does this hold for the rank functions themselves? ) as well as raising some interesting questions (for instance, what is the right notion of bases?).

The exposition is clear and direct, benefiting the author that wants to readily use the important ideas to extend this theory.
It covers Borel partitions and some Borel-preserving operations, Re-randomisation (a method useful to extract integrals), a definition of graphings and the rank function

I also make some typographical comments that are worth correcting.

\section*{Questions}

\begin{itemize}
\entry{5}{Lemma 3.3.} 

\end{itemize}


\section*{Details and comments}


\subsection*{Reading ease comments}

\begin{itemize}
\entry{3}{Theorem 2.3} 

\end{itemize}


\subsection*{Typographical comments}

\begin{itemize}
\entry{1}{The two conjectures then follows}

\end{itemize}

\section*{Recommendation}
I suggest to accept this paper 



\bibliographystyle{alpha}
\bibliography{bibli.bib}

\end{document}
