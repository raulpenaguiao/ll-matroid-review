\title{Review of \say{The matroid of a graphing}}
\date{\today}


\newcommand\entry[2]{\item Page {#1}:  \say{#2} $\rightarrow$}
\def\shuffle{\sqcup\mathchoice{\mkern-7mu}{\mkern-7mu}{\mkern-3.2mu}{\mkern-3.8mu}\sqcup}

\documentclass[12pt]{article}
\usepackage{dirtytalk}
\usepackage{amsmath}
\usepackage{amssymb}
\begin{document}
\maketitle




\section*{Outline}

The paper proposes a notion of cycle matriod for a \say{Benjamini - Schramm} limiting object on graphs, called \textbf{graphings}, prompting a generalisation of a construction of a matroid in limits of graphs.
A notion of infinite matroid is a topic of active research and one can find several proposals in the literature.

This paper tacitly aims at kickstarting a new limit theory on matroids.
It also presents the proof of fundamental claims.
For instance, the  normalized rank functions of a sequence of graphs converges to the normalized rank function of the Benjamini - Schramm limit.
It is also shown that spanning forests on the so called hyperfinite graphings give rise to additive submeasures, which are akin to basis of this generalised matroid.
Finally, it discloses some conjectures.
For instance, the author asks if the limit of does this hold for the rank functions themselves? ) as well as raising some interesting questions (for instance, what is the right notion of bases?).
The author also asks for a full characterisation of bases

The exposition is clear and direct, benefiting the reader that wants to readily use the important ideas to extend this theory.
It covers Borel partitions and some Borel-preserving operations, Re-randomisation (a method useful to extract integrals), a definition of graphings and the rank function.
It also introduces several technical definitions like \textit{locally estimable parameters} \textit{hyperfinite graphings}, \textit{strongly bounded}, \textit{continuous from above}, and \textit{absolutely continuous set functions}.

I also make some typographical comments that are worth correcting.

\section*{Questions}

\begin{itemize}
\entry{15}{minorizing charge for $\rho$ is in fact countably additive} 
I don't know what countably additive means.

\entry{16}{hyperfinite definition} 
Is it possible that this definition can be rephrased as $\eta(\bigcup_{A \text{ infinite connected component}} A) = 0$? Mayeb an example can give a good idea what we are working with


\end{itemize}


\section*{Details and comments}


\subsection*{Reading ease comments}

\begin{itemize}
\entry{3}{Theorem 2.3} 

\end{itemize}


\subsection*{Typographical comments}

\begin{itemize}
\entry{1}{The two conjectures then follows}

\end{itemize}

\section*{Recommendation}
I suggest to accept this paper 



\bibliographystyle{alpha}
\bibliography{bibli.bib}

\end{document}
